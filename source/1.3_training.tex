%%% 1.3 QUALITY OF TRAINING %%%
\subsection{Quality of supervision, training, and knowledge transfer}
\label{sec:training}
\inline{Add some words about Primož}

%The profile of the supervisor is naturally complementary to mine within our proposed research.
During the MCS fellowship in Ljubljana, I will be supervised by prof.\ Primož Potočnik. 
He is a leading expert in the area of algebraic graph theory. His work includes purely graph theoretical
results, applications of group theory in algebraic graph theory, as well as computational aspects in group theory and discrete mathematics.
He has proved a number of deep theoretical results as well as developed many new computational approaches that enabled him (and his coworkers Pablo Spiga, Gabriel Verret) to significantly increase the scope of existing datasets of highly symmetric graphs. His achievements include the census of all
cubic vertex-transitive graphs of order up to $1280$, the census of all tetravalent arc-transitive graphs of order up to $640$, the census of all
rotary maps on at most 1.500 edges etc. These datasets are one of the most used and cited resources in algebraic graph theory.

**** Maybe the following paragraph could be moved to Section 3.2 *****
Prof.\ Potočnik has successfully supervised three PhD students (Gabriel Verret, Katja Berčič, Micael Toledo), currently supervises
a postdoctoral student Alejandra Ramos and has informally advised several junior members of the department. He is the leader of
of a long-term research programme funded by the Slovenian Research Agency (ARRS) has been the head of the PhD programme in mathematics at
University of Ljubljana. He has strong research ties with a number of leading research groups in discrete mathematics (such as the groups at University of Auckland, University of Ottawa, Comenius University in Bratislava, Univerist\'a dagli Studi Milano etc) and regularly hosts top researchers in the area.

Working under his supervision as a MCS fellow, I hope to draw from his vast experience on theoretical and computational issues in algebraic graph theory. In particular, I hope acquire several new skills, including:
\begin{itemize}
\setlength{\itemsep}{0pt}
\item ability of using advanced group theoretical methods relevant for the topic of the proposed research;
\item proficiency in development of software packages (\gap, \sage\ , \magma) for discrete mathematics;
\item internet programming skills, such as construction of user-friendly internet platforms;
\item understanding mathematical knowledge management (presenting and storing mathematical data under FAIR principles).
\end{itemize}

The training and joint research with the supervisor will consists of regular research meetings with the supervisor (at least 4 times a week),
attending the meetings of the departments discrete mathematics group (once a week) and intensive research retreats (at least once per 2 months).
Regular weekly meetings will consist of discussing the weekly plan (on Mondays) at least one long joint research session per week 
and a recap session at the end of the week.


**

While developing the proposed research I will receive training on two different  \emph{theoretical skills} that have been previously used to build datasets of graphs.
We shall explore if and how they could be adapted to developing datasets of polytopes.
On the other hand, to develop the computational part of the project I will acquire new skills on developing and managing datasets.
This skills includes \emph{development of software packages} (\gap, \sage\ , \magma), \emph{database management} (SQL)  and \emph{data publishing} (FAIR repositories, website building, CVS-files manipulation).

On the other hand I am a young researcher on the subject of abstract polytopes with expertise on problems related to building abstract polytopes and similar objects with prescribed combinatorial constraints or imposed symmetry conditions.
FMF-UL is already a leading institution on discrete mathematics and some successful researchers on abstract polytopes and related objects have been formed in this institution in the past. However, there is no currently an active researcher working on this topic.
My presence in FMF-UL will serve as fists steps to revive the area in an already strong environment on discrete mathematics.

The Faculty of Mathematics and Physics of the University of Ljubljana (FMF-UL) has a strong group on computer science who \tero{Say something about how they will interact, drop some names?}.
In particular, Dr. Katja Berčič is an expert on \emph{mathematical knowledge management} and will serve as a consultant for the computational part of the development of the project.

\todo{Mention external committee here?}
% At a minimum, address the following aspects:

%     • Describe the qualifications and experience of the supervisor(s). Provide information regarding the supervisors' level of experience on the research topic proposed and their track record of work, including main international collaborations, as well as the level of experience in supervising/training, especially at advanced level (i.e. PhD and postdoctoral researchers).
%     • Planned training activities for the researcher (scientific aspects, management/organisation, horizontal and key transferrable skills...).
%     • For European Fellowships: two-way transfer of knowledge between the researcher and host organisation.
%     • For Global Fellowships: three-way transfer of knowledge between the researcher, host organisation, and associated partner for outgoing phase.
%     • Rationale and added-value of the non-academic placement (if applicable).
%
% Supervision
% Employers and/or funders should ensure that a person is clearly identified to whom researchers can refer for the performance of their professional duties, and should inform the researchers accordingly.
% Such arrangements should clearly define that the proposed supervisors are sufficiently expert in supervising research, have the time, knowledge, experience, expertise and commitment to be able to offer the research doctoral candidate appropriate support and provide for the necessary progress and review procedures, as well as the necessary feedback mechanisms.
%
%  Supervision is one of the crucial elements of successful research. Guiding, supporting, directing, advising and mentoring are key factors for a researcher to pursue his/her career path. In this context, all MSCA-funded projects are encouraged to follow the recommendations outlined in the MSCA Guidelines on Supervision.


%%% 1.3.1 SUPERVISOR %%%
% \marginLeft{Supervisor}%
% Lorem ipsum dolor sit amet, consectetur adipiscing elit, sed do eiusmod tempor incididunt ut labore et dolore magna aliqua. Ut enim ad minim veniam, quis nostrud exercitation ullamco laboris nisi ut aliquip ex ea commodo consequat. Duis aute irure dolor in reprehenderit in voluptate velit esse cillum dolore eu fugiat nulla pariatur. Excepteur sint occaecat cupidatat non proident, sunt in culpa qui officia deserunt mollit anim id est laborum.
%
%
% %%% 1.3.2 GROUP %%%
% \colorrule
% \marginLeft{Group}%
% Sed ut perspiciatis unde omnis iste natus error sit voluptatem accusantium doloremque laudantium, totam rem aperiam, eaque ipsa quae ab illo inventore veritatis et quasi architecto beatae vitae dicta sunt explicabo. Nemo enim ipsam voluptatem quia voluptas sit aspernatur aut odit aut fugit, sed quia consequuntur magni dolores eos qui ratione voluptatem sequi nesciunt. Neque porro quisquam est, qui dolorem ipsum quia dolor sit amet, consectetur, adipisci velit, sed quia non numquam eius modi tempora incidunt ut labore et dolore magnam aliquam quaerat voluptatem. Ut enim ad minima veniam, quis nostrum exercitationem ullam corporis suscipit laboriosam, nisi ut aliquid ex ea commodi consequatur? Quis autem vel eum iure reprehenderit qui in ea voluptate velit esse quam nihil molestiae consequatur, vel illum qui dolorem eum fugiat quo voluptas nulla pariatur?
%
% %%% 1.3.3 KNOWLEDGE TRANSFER %%%
% \colorrule
% \marginLeft{Knowledge transfer}%
% At vero eos et accusamus et iusto odio dignissimos ducimus qui blanditiis praesentium voluptatum deleniti atque corrupti quos dolores et quas molestias excepturi sint occaecati cupiditate non provident, similique sunt in culpa qui officia deserunt mollitia animi, id est laborum et dolorum fuga. Et harum quidem rerum facilis est et expedita distinctio. Nam libero tempore, cum soluta nobis est eligendi optio cumque nihil impedit quo minus id quod maxime placeat facere possimus, omnis voluptas assumenda est, omnis dolor repellendus.
%
% Temporibus autem quibusdam et aut officiis debitis aut rerum necessitatibus saepe eveniet ut et voluptates repudiandae sint et molestiae non recusandae. Itaque earum rerum hic tenetur a sapiente delectus, ut aut reiciendis voluptatibus maiores alias consequatur aut perferendis doloribus asperiores repellat.
% \textsl{Itaque earum rerum hic tenetur a sapiente delectus, ut aut reiciendis voluptatibus maiores alias consequatur aut perferendis doloribus asperiores repellat.}
%
%
%
