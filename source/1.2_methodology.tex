%%% 1.2 METHODOLOGY %%%
\subsection{Soundness of the proposed methodology (including interdisciplinary approaches, consideration of the gender dimension and other diversity aspects if relevant for the research project, and the quality of open science practices, including sharing and management of research outputs and engagement of citizens, civil society and end users, where appropriate)}
\label{sec:methodology}
% At a minimum, address the following aspects:
%     • Overall methodology: Describe and explain the overall methodology, including the concepts, models and assumptions that underpin your work. Explain how this will enable you to deliver your project’s objectives. Refer to any important challenges you may have identified in the chosen methodology and how you intend to overcome them.
%
%     • Integration of methods and disciplines to pursue the objectives: Explain how expertise and methods from different disciplines will be brought together and integrated in pursuit of your objectives. If you consider that an inter-disciplinary1 approach is unnecessary in the context of the proposed work, please provide a justification.
%     • Gender dimension and other diversity aspects: Describe how the gender dimension and other diversity aspects are taken into account in the project’s research and innovation content. If you do not consider such a gender dimension to be relevant in your project, please provide a justification.
%     • Remember that that this question relates to the content of the planned research and innovation activities, and not to gender balance in the teams in charge of carrying out the project.
%     • Sex, gender and diversity analysis refers to biological characteristics and social/cultural factors respectively. For guidance on methods of sex / gender analysis and the issues to be taken into account, please refer to this page.
%
% Open science practices: Describe how appropriate open science practices are implemented as an integral part of the proposed methodology. Show how the choice of practices and their implementation is adapted to the nature of your work in a way that will increase the chances of the project delivering on its objectives [e.g. up to 1/2 page, including research data management]. If you believe that none of these practices are appropriate for your project, please provide a justification here.
%
% Open science is an approach based on open cooperative work and systematic sharing of knowledge and tools as early and widely as possible in the process. Open science practices include early and open sharing of research (for example through pre-registration, registered reports, pre-prints, or crowd-sourcing); research output management; measures to ensure reproducibility of research outputs; providing open access to research outputs (such as publications, data, software, models, algorithms, and workflows); participation in open peer-review; and involving all relevant knowledge actors including citizens, civil society and end users in the co-creation of R&I agendas and contents (such as citizen science).
%
%     • Please note that this does not refer to outreach actions that may be planned as part of the communication, dissemination and exploitation activities. These aspects should instead be described below under ‘Impact’.
%
%     • Research data management and management of other research outputs: Applicants generating/collecting data and/or other research outputs (except for publications) during the project must explain how the data will be managed in line with the FAIR principles (Findable, Accessible, Interoperable, Reusable).
%     • For guidance on open science practices and research data management, please refer to the relevant section of the HE Programme Guide on the Funding & Tenders Portal.

The nature of our project involves a two way flow of knowledge between from discrete mathematics and group theory to the development of computational tools and data management. In order to build new data sets of abstract polytopes we need computational-efitient ways to compute and storage property of such objects and the other way around, if we want to OURPROJECT to eventually becomes a useful tool on theoretical research we need to develop ways to access and present the computed information in a user friendly way. We describe our proposed metodology from this two complementary approaches.

Mathematical representation of highly symmetric abstract polytopes.

Abrstract polytopes as posets. The original definition of an abstract polytope is on the form of a poset [REF]. It makes sense from an hystorical view point. They are intendet to be combinatorial generalisation of the (geometric) convex polytopes. This generalisation was obtained by taking some of the properties of the face lattice of a convex polytope and use them as defining properties of an abstract polytope. Unfortunatelly, the computational cost and the combinatorial problem of storaging a poset seems to be very inneficient. Moreover, modern computer algebra systems do not have particullarly efficient tools to deal with posets. We should try to avoid representing abstract polytopes as parially ordered sets.

Abstract polytopes from their automorphism group. Most likely this is the most exploided representation of an abstract polytopes. When an abstract polytope has a high degree of symmetry its automorphism group contains many combinatorial information of the poltyope. In particular, regular polytopes are incorrespondence with string C-groups [REF!], which are smoot quotient of Coxeter groups satisfying certain interection property. This fact has been strongly used to build the existing datasets mentioned in Section 1.1. The census [CONDER] was built by computing all possible normal subgroups of index at most 2000 of the universal string Coxeter group. This approach is computational expensive but it has the advantage that the computations have to be done only once. It might be worthy to try an push


%%% 1.2.1 METHODS %%%


%%% 1.2.2 METHODS vs OBJECTIVES %%%
% \colorrule
% \marginLeft{Approach}%
%
% \fbox{WP1} Sed ut \highlight{D1.1} perspiciatis unde omnis iste natus error sit voluptatem accusantium doloremque
