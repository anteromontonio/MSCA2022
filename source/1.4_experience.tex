%%% 1.4 POTENTIAL OF THE RESEARCHER %%%
\subsection{Quality and appropriateness of the researcher’s professional experience, competences and skills}
\label{sec:experience}

% Discuss the quality and appropriateness of the researcher’s existing professional experience in relation to the proposed research project.

The candidate is a young researcher with wide experience on abstract polytopes.
He has a strong background on several topics of discrete mathematics, but in particular on abstract polytopes.
He started participating on conferences and workshops more than 10 years ago and has been active on the community ever since.
The early years of his academic career focused on the geometric side of abstract polytopes. He started doing research on abstract polytopes very early in his career. His undergraduate thesis\footcite{Montero_2013_PoliedrosRegularesEn} attacks the problem of enumerating toroidal polyhedra.
Later on he did a complete classification of such polyhedra\footcite{Montero_2018_RegularPolyhedra3}.
During his Ph. D. the candidate worked under the supervision of Daniel Pellicer, one of the young leading experts on abstract polytopes.
The candidate moved his research interests to one of the most challenging problems on the theory: constructing chiral polytopes with prescribed regular facets.
The results obtained\footcite{Montero_2019_ChiralExtensionsToroids_PhDThesis}\footcite{MonteroPellicerToledo__ChiralExtensionsRegular_preprint} were only partial but trained the candidate on several approaches and the process gave him tools from geometry, combinatorics, group theory and programming.

The candidate did a Postdoctoral visit of one year in York University (Canada) under the supervision of A. Weiss, one of the most established researchers on the area were the research focus was mostly on building highly symmetric polytopal objects\footcite{MonteroWeiss_2021_LocallySphericalHypertopes}\footcite{MonteroWeiss_2021_ProperLocallySpherical}.
He continued his career in the National Autonomous University of Mexico, where he spent one year and a half working with I. Hubard.
During this period he continued his work on building symmetrical objects.
In particular, his research turn into the problem of building maniplexes and polytopes with given symmetry type. The candidate started two projects on this topic: One related to operations of maniplexes with Hubard and Mochán and another related to extensions of maniplexes in joint work with G. Cunningham and Mochán.

His academic relation with the Slovenian mathematical community began when he was a Ph. D. student in 2017 and spent a research visit of six months in the University of Ljubljana.

The candidate has experience not only doing research but also on the communications of mathematics.
He has given more than 35 talks on colloquia, seminars and conferences, including some of them as invited speaker.
In 2018 the candidate was awarded with the \emph{best student talk award} in the $8th$ Ph.D. summer school in discrete mathematics.
The candidate has also participated and organised outreach activities, which has given them some team management skills.
