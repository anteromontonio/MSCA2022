%%% 2.2 QUALITY OF PROPOSED MEASURES %%%
%Suitability and quality of the measures to maximise expected outcomes and impacts, as set out in the dissemination and exploitation plan, including communication activities
\subsection{Suitability and quality of the measures to maximise expected outcomes and impacts, as set out in the dissemination and exploitation plan, including communication activities}
\label{sec:suitability}

As described in previous section, our project has two main branches: a purely theoretical one and a computational one. The activities planned to disseminate and communicate our activities can be naturally divided in those two branches as well.

For the theoretical part of our research the candidate will have day-to-day conversations not only with the supervisor but also with other researches on the math department of FMF-UL. The candidate will constantly participate on the \emph{discrete mathematics seminar} of FMF-UL and the \emph{combinatorics and group theory seminar} of the Faculty of Education of the University of Ljubljana (PeF-UL). The candidate will participate on the 2023 International Slovenian Conference on Graph theory, which is an international forum that occurs every 4 years and has had more than 300 participants in the last editions.
These activities align perfectly with \cref{obj:theoreticalConst,obj:usage}.


In other to develop the computational part of the project and the actual development of \cref{obj:buildNew}, the candidate will work in close contact with Katja Berčič, who is an expert on \emph{mathematical knowledge management} and the creation of datasets.
The candidate will organise at least one \emph{programming workshop} per year where both researcher and students will get together and discuss problems regarding datasets of discrete objects.

The research objects involved in \ourp are not only of mathematical interest but the have a natural degree of beauty. With this in mind the candidate is expected to participate in the $28th$ and $29th$ editions of the Slovenian Festival of Science, where some of the aspects of the research developed can be shared to a general audience with activities such as \emph{Origami Polyhedra} or \emph{Build your own kaleidoscope}.


%
% At a minimum, address the following aspects:
% Plan for the dissemination and exploitation activities, including communication activities:1 Describe the planned measures to maximize the impact of your project by providing a first version of your ‘plan for the dissemination and exploitation including communication activities’. Describe the dissemination, exploitation measures that are planned, and the target group(s) addressed (e.g. scientific community, end users, financial actors, public at large). Regarding communication measures and public engagement strategy, the aim is to inform and reach out to society and show the
% activities performed, and the use and the benefits the project will have for citizens. Activities must be strategically planned, with clear objectives, start at the outset and continue through the lifetime of the project. The description of the communication activities needs to state the main messages as well as the tools and channels that will be used to reach out to each of the chosen target groups.
%     • Strategy for the management of intellectual property, foreseen protection measures: if relevant, discuss the strategy for the management of intellectual property, foreseen protection measures, such as patents, design rights, copyright, trade secrets, etc., and how these would be used to support exploitation.
%
%     • All measures should be proportionate to the scale of the project, and should contain concrete actions to be implemented both during and after the end of the project.



%%% 2.2.1 DISSEMINATION %%%
% \colorrule
% \marginLeft{Dissemination}%
% Sed ut perspiciatis unde omnis iste natus error sit voluptatem accusantium doloremque laudantium, totam rem aperiam, eaque ipsa quae ab illo inventore veritatis et quasi architecto beatae vitae dicta sunt explicabo. Nemo enim ipsam voluptatem quia voluptas sit aspernatur aut odit aut fugit, sed quia consequuntur magni dolores eos qui ratione voluptatem sequi nesciunt. Neque porro quisquam est, qui dolorem ipsum quia dolor sit amet, consectetur, adipisci velit, sed quia non numquam eius modi tempora incidunt ut labore et dolore magnam aliquam quaerat voluptatem. Ut enim ad minima veniam, quis nostrum exercitationem ullam corporis suscipit laboriosam, nisi ut aliquid ex ea commodi consequatur? Quis autem vel eum iure reprehenderit qui in ea voluptate velit esse quam nihil molestiae consequatur, vel illum qui dolorem eum fugiat quo voluptas nulla pariatur?
%
% %%% 2.2.2 COMMUNICATION %%%
% \colorrule
% \marginLeft{Communication}%
%
%
%
% %%% 2.2.3 STRATEGY %%%
% \colorrule
% \marginLeft{Strategy}%
