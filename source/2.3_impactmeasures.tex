%%% 2.3 IMPACT ON SCIENCE, ECONOMY, and SOCIETY %%%
%The magnitude and importance of the project’s contribution to the expected scientific, societal and economic impacts
\subsection{The magnitude and importance of the project’s contribution to the expected scientific, societal and economic impacts}
\label{sec:impactmeasures}

As mentioned before, the final users of our project are mostly on the scientific community.

Our research shall contribute to the overall development of research on symmetries of discrete objects.
As mentioned in \cref{sec:suitability}, \ourp\ shall benefit researchers and students of discrete mathematics by improving the way the conduct research. The first and probably most obvious way is that as a result of \cref{obj:datasets} and \cref{obj:theory} we shall not only contribute with research papers to the community but to show an example of how datasets can be used to conduct research.
Our data will be published as an open FAIR-repository with the appropriate documentation and software packages so that the users can easily include it on their own workflow.
Furthermore, with the success of \cref{obj:longterm} we expect to truly and substantially change the way research on discrete mathematics is performed.

The collaboration with the KWARC group will serve to promote the development of mathematical knowledge management. In particular, our project, as a whole, shall be a test case of the Deep FAIR principles.


%
%     • Provide a narrative explaining how the project’s results are expected to make a difference in terms of impact, beyond the immediate scope and duration of the project. The narrative should include the components below, tailored to your project.
%     • Be specific, referring to the effects of your project, and not R&I in general in this field. State the target groups that would benefit.
%     • Expected scientific impact(s): e.g. contributing to specific scientific advances, across and within disciplines, creating new knowledge, reinforcing scientific equipment and instruments, computing systems (i.e. research infrastructures);
%     • Expected economic/technological impact(s): e.g. bringing new products, services, business processes to the market, increasing efficiency, decreasing costs, increasing profits, contributing to standards’ setting, etc.
%     • Expected societal impact(s): e.g. decreasing CO2 emissions, decreasing avoidable mortality, improving policies and decision-making, raising consumer awareness.
%     • Only include such outcomes and impacts where your project would make a significant and direct contribution. Avoid describing very tenuous links to wider impacts.
%
%     • Give an indication of the magnitude and importance of the project’s contribution to the expected outcomes and impacts, should the project be successful. Provide quantified estimates where possible and meaningful. ‘Magnitude’ refers to how widespread the outcomes and impacts are likely to be. For example, in terms of the size of the target group, or the proportion of that group, that should benefit over time; ‘Importance’ refers to the value of those benefits. For example, number of additional healthy life years; efficiency savings in energy supply.

%%% 2.3.1 Science %%%



